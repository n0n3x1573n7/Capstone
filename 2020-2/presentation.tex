% !TeX document-id = {9fe05573-0d66-48b0-a749-e436b3eb22be}
%!TeX TXS-program:compile = txs:///xelatex/
\documentclass{beamer}
\usetheme{Darmstadt}

\newif\ifproposal
\proposalfalse

\usepackage{standalone}

\usepackage{CJKutf8}

\usepackage[utf8]{inputenc}
\usepackage[OT1, T2A]{fontenc}
\usepackage{fontspec}
\setmainfont[ItalicFont={*},ItalicFeatures={FakeSlant=.167}]{D2Coding}

\usepackage[normalem]{ulem}

\usepackage{graphicx}
\usepackage{tikz}
%\usetikzlibrary{quantikz}
\usepackage{braket}
\newcommand{\iu}{{i\mkern1mu}}

% based on the original definitions in beamerbasenavigation.sty
\makeatletter
\def\sectionentry#1#2#3#4#5{% section number, section title, page
    %
    \newcount\mymin%
    \mymin=3
    \ifnum\c@section=1%
    \mymin=5
    \fi%
    \ifnum\c@section=2%
    \mymin=4
    \fi%
    %
    \newcount\mymax%
    \mymax=3
    \ifnum\c@section=\beamer@sectionmax%
    \mymax=5
    \fi%
    \ifnum\c@section=\numexpr\beamer@sectionmax-1%
    \mymax=4
    \fi%
    %
    \ifnum\numexpr\c@section-#1<\mymax%
    \ifnum\numexpr#1-\c@section<\mymin%
    \ifnum#5=\c@part%
    \beamer@section@set@min@width
    \box\beamer@sectionbox\hskip1.875ex plus 1fill%
    \beamer@xpos=0\relax%
    \beamer@ypos=1\relax%
    \setbox\beamer@sectionbox=
    \hbox{
        \def\insertsectionhead{#2}%
        \def\insertsectionheadnumber{#1}%
        \def\insertpartheadnumber{#5}%

        {%
            \usebeamerfont{section in head/foot}\usebeamercolor[fg]{section in head/foot}%
            \ifnum\c@section=#1%
            \hyperlink{Navigation#3}{{\usebeamertemplate{section in head/foot}}}%
            \else%
            \hyperlink{Navigation#3}{{\usebeamertemplate{section in head/foot shaded}}}%
            \fi%
        }%
    }%
    \ht\beamer@sectionbox=1.875ex%
    \dp\beamer@sectionbox=0.75ex%
    \fi%
    \fi%
    \fi%
    \ignorespaces%
}

\def\slideentry#1#2#3#4#5#6{%
    %section number, subsection number, slide number, first/last frame, page number, part number
    %
    \newcount\mymin%
    \mymin=3
    \ifnum\c@section=1%
    \mymin=5
    \fi%
    \ifnum\c@section=2%
    \mymin=4
    \fi%
    %
    \newcount\mymax%
    \mymax=3
    \ifnum\c@section=\beamer@sectionmax%
    \mymax=5
    \fi%
    \ifnum\c@section=\numexpr\beamer@sectionmax-1%
    \mymax=4
    \fi%
    %
    \ifnum\numexpr\c@section-#1<\mymax%
    \ifnum\numexpr#1-\c@section<\mymin%
    \ifnum#6=\c@part\ifnum#2>0\ifnum#3>0%
    \ifbeamer@compress%
    \advance\beamer@xpos by1\relax%
    \else%
    \beamer@xpos=#3\relax%
    \beamer@ypos=#2\relax%
    \fi%
    \hbox to 0pt{%
        \beamer@tempdim=-\beamer@vboxoffset%
        \advance\beamer@tempdim by-\beamer@boxsize%
        \multiply\beamer@tempdim by\beamer@ypos%
        \advance\beamer@tempdim by -.05cm%
        \raise\beamer@tempdim\hbox{%
            \beamer@tempdim=\beamer@boxsize%
            \multiply\beamer@tempdim by\beamer@xpos%
            \advance\beamer@tempdim by -\beamer@boxsize%
            \advance\beamer@tempdim by 1pt%
            \kern\beamer@tempdim
            \global\beamer@section@min@dim\beamer@tempdim
            \hbox{\beamer@link(#4){%
                    \usebeamerfont{mini frame}%
                    \ifnum\c@section=#1%
                    \ifnum\c@subsection=#2%
                    \usebeamercolor[fg]{mini frame}%
                    \ifnum\c@subsectionslide=#3%
                    \usebeamertemplate{mini frame}%\beamer@minislidehilight%
                    \else%
                    \usebeamertemplate{mini frame in current subsection}%\beamer@minisliderowhilight%
                    \fi%
                    \else%
                    \usebeamercolor{mini frame}%
                    %\color{fg!50!bg}%
                    \usebeamertemplate{mini frame in other subsection}%\beamer@minislide%
                    \fi%
                    \else%
                    \usebeamercolor{mini frame}%
                    %\color{fg!50!bg}%
                    \usebeamertemplate{mini frame in other subsection}%\beamer@minislide%
                    \fi%
        }}}\hskip-10cm plus 1fil%
    }\fi\fi%
    \else%
    \fakeslideentry{#1}{#2}{#3}{#4}{#5}{#6}%
    \fi%
    \fi%
    \fi%
    \ignorespaces%
}
\makeatother

\AtBeginSection[]{
    \begin{frame}
        \vfill
        \centering
        \begin{beamercolorbox}[sep=8pt,center,shadow=true,rounded=true]{title}
            \usebeamerfont{title}\insertsectionhead\par%
        \end{beamercolorbox}
        \vfill
    \end{frame}
}

\usepackage{xcolor}
\usepackage{pgfplots}
\usepackage{tikz}

\setbeamertemplate{bibliography entry title}{}
\setbeamertemplate{bibliography entry location}{}
\setbeamertemplate{bibliography entry note}{}

% Define bar chart colors
%
\definecolor{bblue}{HTML}{4F81BD}
\definecolor{rred}{HTML}{C0504D}
\definecolor{ggreen}{HTML}{9BBB59}
\definecolor{ppurple}{HTML}{9F4C7C}


\title{Optimizing Quantum Oracle for Standard Hash Algorithms}
%\subtitle{Week 7 Report for Class 75}

\author{Korea University\\Sangheon Lee, Mingyu Cho}

\date{\today}

\begin{document}
    \begin{frame}
        \titlepage
    \end{frame}

    \begin{frame}
        \frametitle{Table of Contents}
        \tableofcontents
    \end{frame}
    
    \section{Introduction}
    \begin{frame}{Previous Work}
        \begin{itemize}
            \item Studied the basics of quantum computing and Grover's Algorithm
            \item Analysis of reversible quantum S-DES oracle and its construction
            \item Implementation of quantum S-DES oracle with Grover's Algorithm in Microsoft Q\#
        \end{itemize}
    \end{frame}
    
    \begin{frame}{Motivation}
        % why SHA-2/3? why preimage attack (or collision attack, nvm)
        \begin{itemize}
          \item Targeting SHA-2/3
          \begin{itemize}
            \item Current de facto standard hash algorithm
          \end{itemize}
          \item Goal is to perform a preimage attack.
          \begin{itemize}
            \item The most straightforward to apply Grover's Algorithm.
            \item It will be assumed that Grover's Algorithm(or some variation of it) will be applied
            \item Therefore target is to optimize the quantum oracle for SHA-2/3.
          \end{itemize}
        \end{itemize}
    \end{frame}
    
    \begin{frame}{Recap: Grover's Algorithm}
        \begin{itemize}
            \item Searching in an unordered database among $2^n$ elements takes $\tilde{\mathcal{O}}(2^{n/2})$ time complexity and $\tilde{\mathcal{O}}(n)$ quantum space complexity
            \item Proved to be optimal in general
            \item Specifically, improved Grover's Algorithm for collision search yields $\tilde{\mathcal{O}}(2^{n/3})$ time complexity and (quantum) space complexity \cite{brassard1998quantum}
            \begin{itemize}
                \item However, quantum queries are costly in this algorithm
            \end{itemize}
        \end{itemize}
    \end{frame}
    
    \begin{frame}{Chailloux's Algorithm \cite{chailloux2017efficient}}
        \begin{itemize}
            \item Use poly-time qubits and reduce time complexity to $\tilde{\mathcal{O}}(2^{2n/5})$ for collision search and $\tilde{\mathcal{O}}(2^{3n/7})$ for multi-target preimage attacks with additional classic memory
        \end{itemize}
    \end{frame}
    
    \begin{frame}{SHA-2/3 Pre-image Attack Cost Estimation \cite{amy2016estimating}}
        \begin{itemize}
            \item Suggest cost metric for quantum computation based on surface code
            \item Theoretically implement reversible SHA-2/3 quantum circuits
            \item Estimate required physical resources and its scale
        \end{itemize}
    \end{frame}
    
    \begin{frame}{Expectations}
      \begin{itemize}
        \item Focus on micro-scale improvement(regarding quantum oracle)
        \item May be able to reduce the cost in at least one of the following metrics:
          \begin{itemize}
            \item The cost metric from before
            \item The raw number of gates
            \item The number of levels after achieving parallelization
          \end{itemize}
        \item Main target will be the cost metric, not the gates or the levels
      \end{itemize}
    \end{frame}
    
    \section{In-depth Topics}
    \begin{frame}{Topics}
        \begin{itemize}
            \item Surface code(Toric code) : topological quantum error correcting code
            \item Cost metric based on surface code
            \item T-par \cite{amy2014polynomial} : an quantum circuit optimization tool
            \item Advanced quantum circuit (in-place adder, \textit{etc}.)
        \end{itemize}
    \end{frame}
    
    \begin{frame}[allowframebreaks]
        \frametitle{References}
        \bibliographystyle{amsalpha}
        \bibliography{./quantum.bib}
    \end{frame}
\end{document}