\documentclass{beamer}
\usetheme{Darmstadt}
\usepackage{CJKutf8}

\usepackage[utf8]{inputenc}
\usepackage[OT1, T2A]{fontenc}
\usepackage{fontspec}
\usepackage[normalem]{ulem}

\usepackage{graphicx}
\usepackage{tikz}
\usetikzlibrary{quantikz}
\usepackage{braket}

\usepackage{xcolor}
\usepackage{pgfplots}
\usepackage{tikz}

\begin{document}
	\begin{frame}{Overview}
		\begin{itemize}
			\item Most part can be expressed in permutations, but there are parts where we need to use quantum gates.
			\item In S-Box, we inevitably need to use Pauli-X and Toffoli gates.
			\item In EP, we need to copy the bits, and thereby requires the use of Toffoli gates.
			\item In round application, we need to XOR the subkey with some data or S-box result, requiring us to use CNOT gates.
		\end{itemize}
	\end{frame}
	
	\begin{frame}{Number of Gates: S-Box}
		\begin{itemize}
			\item Both S-box requires certain number of Pauli-X and Toffoli(CCNOT) gates.
			\item The sum of gates required for each of the 16 inputs possible:
			\begin{center}
				\begin{tabular}{c|c|c}
					        & Pauli-X & Toffoli \\\hline
					S-box 1 & 50      & 18      \\\hline
					S-box 2 & 44      & 15
				\end{tabular}
			\end{center}
			\item On average, \textasciitilde3 Pauli-X gates and \textasciitilde1 Toffoli gate is used.
		\end{itemize}
	\end{frame}
	
	\begin{frame}{Number of Gates: S-Box}
		\begin{itemize}
			\item What about Quine-McCluskey Method?
			\item Looking at S-box 1:
			\begin{itemize}
				\item S1 bit 1: $AB'C'+A'B'C+BCD'+ABD+BC'D$
				\item S1 bit 2: $AD'+B'D'+A'BC+ABC'$
			\end{itemize}
			\item S1 requires 7 OR operations, all of which must be done with Toffoli gates.
			\item Compared to the brute-force method, this takes even more gates!
		\end{itemize}
	\end{frame}
	
	\begin{frame}{Number of Gates: S-Box Input and Output}
		\begin{itemize}
			\item Assume 3 Pauli-X gates and 1 Toffoli gate per single S-box calculation.
			\item For input, 4 CNOT gates are required.
			\item To copy the output, 2 more CNOT gates are required.
			\item To make this reversible, S-box application and input processes must be reversed, requiring 2 S-Box applications and 8 CNOT gates.
			\item[$\Rightarrow$] Total of 10 CNOT gates, 6 Pauli-X gates, and 2 Toffoli gates.
		\end{itemize}
	\end{frame}
	
	\begin{frame}{Number of Gates: Apply Rounds}
		\begin{itemize}
			\item Two S-Boxes are applied per round.
			\begin{itemize}
				\item The remaining operations are all permutations.
			\end{itemize}
			\item There are two rounds in total.
			\item[$\Rightarrow$] Total of 40 CNOT gates, 24 Pauli-X gates, and 8 Toffoli gates per single oracle call!
		\end{itemize}
	\end{frame}
\end{document}