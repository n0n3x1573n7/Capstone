\documentclass{beamer}
\usetheme{Darmstadt}
\usepackage{CJKutf8}

\usepackage[utf8]{inputenc}
\usepackage[OT1, T2A]{fontenc}
\usepackage{fontspec}
\usepackage[normalem]{ulem}

\usepackage{graphicx}
\usepackage{tikz}
\usetikzlibrary{quantikz}
\usepackage{braket}

\usepackage{xcolor}
\usepackage{pgfplots}
\usepackage{tikz}

\begin{document}
    \begin{frame}{Prerequisites}
        \begin{itemize}
            \item Let $ SDES(P, K) = C $ be a function s.t. encrypting plaintext $ P $ with S-DES and key $ K $ yields $ C $.
            \item Let $ f : \{0, 1\}^8 \times \{0, 1\}^8 \to \mathcal{P}({\{0, 1\}^{10}})$ be a function such that $ f(P, C) = \{ K \in \{0, 1\}^{10} \,\vert\, SDES(P, K) = C\}$.
            \item Since key size is 4x of cipher size, we guessed that for arbitrary $ P $ and $ C $, $ \left\vert f(P, C) \right\vert$ is likely to be 4.
        \end{itemize}
    \end{frame}

    \begin{frame}{Prerequisites : Key Size}
        ...and we were wrong!
        % include graph here
        \begin{tikzpicture}
            \begin{axis}[
                width  = 0.95*\textwidth,
                height = 0.8\textheight,
%                major x tick style = transparent,
                ybar,
                bar width=6pt,
                ymajorgrids = true,
                ylabel = {\# of $ (P, C) $ pairs},
                xlabel = {Key size},
                symbolic x coords={1,2,3,4,5,6,7,8,9,10,11,12,13,14,15,16,17},
                xtick = data,
                enlarge x limits=0.05,
                ymin=0,
                ymode=log,
            ]

                \addplot[style={bblue,fill=bblue,mark=none}]
                    coordinates {
                        (1, 4376)
                        (2, 10696)
                        (3, 6888)
                        (4, 11896)
                        (5, 3848)
                        (6, 9576)
                        (7, 2728)
                        (8, 4184)
                        (9, 968)
                        (10, 1944)
                        (11, 272)
                        (12, 456)
                        (13, 16)
                        (14, 24)
                        (15, 96)
                        (16, 8)
                        (17, 8)
                        };
            \end{axis}
        \end{tikzpicture}
    \end{frame}

    \begin{frame}{Quantum S-DES Testing : Testing with \#Key = 1}
        If $ (P, C) =$ (\texttt{10100100}, \texttt{00110011}). Then $ f(P, C) = $  $\{$\texttt{1101100110}$\}$.
        \begin{itemize}
            \item Program yielded correct key.
            \item By dumping key qubits, we can observe the probability of each basis.
            \begin{itemize}
                \item At first, the probability of answer basis is $ 0.008766 $ while the others are $ 0.000969 $.
                \item The gap widens next turn: $ 0.024224 $ and $ 0.000954 $.
                \item Ultimately it becomes $ 0.999461 $ and $ 0.000001 $ (at 25th iteration).
            \end{itemize}
        \end{itemize}
    \end{frame}


    \begin{frame}{Quantum S-DES Testing : Testing with \#Key = 4}
        If $ (P, C) =$ (\texttt{11000111}, \texttt{00010100}). Then $ \vert f(P, C) \vert = 4$.
        \begin{itemize}
            \item Program yielded wrong key.
            \item By dumping key qubits, we can observe the probability of each basis.
            \begin{itemize}
                \item At first, the probability of answer basis is $ 0.008698 $ while the others are $ 0.000946 $.
                \item The gap widens next turn: $ 0.023659 $ and $ 0.000888 $.
                \item It peaks at 12th iteration: $ 0.249987 $ and $ 0.000000 $.
                \item However it goes back next turn: $ 0.246547 $ and $ 0.000014 $
                \item Ultimately the probability of answer basis becomes lower than the others: $ 0.000575 $ and $ 0.000978 $.
            \end{itemize}
            \item Although the algorithm and the implementation is correct, what happened here? % 논문 추가
        \end{itemize}
    \end{frame}

    \begin{frame}{Quantum S-DES Testing: What Went Wrong?}
        \begin{itemize}
            \item When running Grover's Algorithm, there is a number of desired steps, say $n$.
            \item If we overshoot that number of step, the probability of basis decreases!
            \item On the $2n$-th step, the probability becomes all the same for each of the cases again, forming a cycle.
        \end{itemize}
    \end{frame}
\end{document}