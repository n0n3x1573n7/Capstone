\documentclass{beamer}
\usetheme{Darmstadt}
\usepackage{CJKutf8}

\usepackage[utf8]{inputenc}
\usepackage[OT1, T2A]{fontenc}
\usepackage{fontspec}
\usepackage[normalem]{ulem}

\usepackage{graphicx}
\usepackage{tikz}
\usetikzlibrary{quantikz}
\usepackage{braket}

\usepackage{xcolor}
\usepackage{pgfplots}
\usepackage{tikz}

\begin{document}
\begin{frame}
        \frametitle{Work in Progress(Mingyu Cho)}
        \begin{itemize}
            \item Implemented a non-quantum S-DES(in python) for testing the quantum version.
            \item Implemented Brute-force attack on S-DES and developed a code for keycount for (plaintext, ciphertext) pair.
            \item Started preliminary analysis on the gate applications
            \item Looking for possible speedups on the quantum side to aid the simulation
        \end{itemize}
    \end{frame}

    \begin{frame}
        \frametitle{Work in Progress(Sangheon Lee)}
        \begin{itemize}
            \item Possibly reduce some quantum gates
            \item Analyze complexity of each gate
            \item Search for internal dumping library
        \end{itemize}
    \end{frame}

    \begin{frame}
        \frametitle{Future Plans}
        \begin{itemize}
            %\item Dig in deeper on the backgrounds, if necessary.
            %\item Understand Grover's Algorithm and other quantum algorithms to know how to utilize it.
            %\item Update sample code of Grover's Algorithm by implementing oracle function and randomization.
            %\item Possibly utilize IBM Quantum Experience?
            %\item Search some more sample codes (not sure if implementing and debugging Shor's and QFT is necessary).
            \item Construct and test S-DES and apply Grover's.
            \begin{itemize}
                \item Since S-DES consists of various components, all of these must be implemented carefully and in order
            \end{itemize}
            \item Hope to import Microsoft Q\# implementation to qiskit (IBM Quantum Experience)
            \item Reduce complexity of quantum gates if possible (as Prof. Hong suggested)
        \end{itemize}
    \end{frame}
\end{document}