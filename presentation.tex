% !TeX document-id = {4fba684d-7122-40d6-bb19-80adba5d32d5}
%!TeX TXS-program:compile = txs:///xelatex/
\documentclass{beamer}
\usetheme{Darmstadt}

\newif\ifproposal
\proposalfalse

\usepackage{CJKutf8}

\usepackage[utf8]{inputenc}
\usepackage[OT1, T2A]{fontenc}
\usepackage{fontspec}
\setmainfont[ItalicFont={*},ItalicFeatures={FakeSlant=.167}]{D2Coding}

\usepackage{braket}
\newcommand{\iu}{{i\mkern1mu}}

% based on the original definitions in beamerbasenavigation.sty
\makeatletter
\def\sectionentry#1#2#3#4#5{% section number, section title, page
	%
	\newcount\mymin%
	\mymin=3
	\ifnum\c@section=1%
	\mymin=5
	\fi%
	\ifnum\c@section=2%
	\mymin=4
	\fi%
	%
	\newcount\mymax%
	\mymax=3
	\ifnum\c@section=\beamer@sectionmax%
	\mymax=5
	\fi%
	\ifnum\c@section=\numexpr\beamer@sectionmax-1%
	\mymax=4
	\fi%
	%
	\ifnum\numexpr\c@section-#1<\mymax%
	\ifnum\numexpr#1-\c@section<\mymin%
	\ifnum#5=\c@part%
	\beamer@section@set@min@width
	\box\beamer@sectionbox\hskip1.875ex plus 1fill%
	\beamer@xpos=0\relax%
	\beamer@ypos=1\relax%
	\setbox\beamer@sectionbox=
	\hbox{
		\def\insertsectionhead{#2}%
		\def\insertsectionheadnumber{#1}%
		\def\insertpartheadnumber{#5}%
		
		{%
			\usebeamerfont{section in head/foot}\usebeamercolor[fg]{section in head/foot}%
			\ifnum\c@section=#1%
			\hyperlink{Navigation#3}{{\usebeamertemplate{section in head/foot}}}%
			\else%
			\hyperlink{Navigation#3}{{\usebeamertemplate{section in head/foot shaded}}}%
			\fi%    
		}%
	}%
	\ht\beamer@sectionbox=1.875ex%
	\dp\beamer@sectionbox=0.75ex%
	\fi%
	\fi%
	\fi%
	\ignorespaces%
}

\def\slideentry#1#2#3#4#5#6{%
	%section number, subsection number, slide number, first/last frame, page number, part number
	%
	\newcount\mymin%
	\mymin=3
	\ifnum\c@section=1%
	\mymin=5
	\fi%
	\ifnum\c@section=2%
	\mymin=4
	\fi%
	%
	\newcount\mymax%
	\mymax=3
	\ifnum\c@section=\beamer@sectionmax%
	\mymax=5
	\fi%
	\ifnum\c@section=\numexpr\beamer@sectionmax-1%
	\mymax=4
	\fi%
	%
	\ifnum\numexpr\c@section-#1<\mymax%
	\ifnum\numexpr#1-\c@section<\mymin%
	\ifnum#6=\c@part\ifnum#2>0\ifnum#3>0%
	\ifbeamer@compress%
	\advance\beamer@xpos by1\relax%
	\else%
	\beamer@xpos=#3\relax%
	\beamer@ypos=#2\relax%
	\fi%
	\hbox to 0pt{%
		\beamer@tempdim=-\beamer@vboxoffset%
		\advance\beamer@tempdim by-\beamer@boxsize%
		\multiply\beamer@tempdim by\beamer@ypos%
		\advance\beamer@tempdim by -.05cm%
		\raise\beamer@tempdim\hbox{%
			\beamer@tempdim=\beamer@boxsize%
			\multiply\beamer@tempdim by\beamer@xpos%
			\advance\beamer@tempdim by -\beamer@boxsize%
			\advance\beamer@tempdim by 1pt%
			\kern\beamer@tempdim
			\global\beamer@section@min@dim\beamer@tempdim
			\hbox{\beamer@link(#4){%
					\usebeamerfont{mini frame}%
					\ifnum\c@section=#1%
					\ifnum\c@subsection=#2%
					\usebeamercolor[fg]{mini frame}%
					\ifnum\c@subsectionslide=#3%
					\usebeamertemplate{mini frame}%\beamer@minislidehilight%
					\else%
					\usebeamertemplate{mini frame in current subsection}%\beamer@minisliderowhilight%
					\fi%
					\else%
					\usebeamercolor{mini frame}%
					%\color{fg!50!bg}%
					\usebeamertemplate{mini frame in other subsection}%\beamer@minislide%
					\fi%
					\else%
					\usebeamercolor{mini frame}%
					%\color{fg!50!bg}%
					\usebeamertemplate{mini frame in other subsection}%\beamer@minislide%
					\fi%
		}}}\hskip-10cm plus 1fil%
	}\fi\fi%
	\else%
	\fakeslideentry{#1}{#2}{#3}{#4}{#5}{#6}%
	\fi%
	\fi%
	\fi%
	\ignorespaces%
}
\makeatother

\AtBeginSection[]{
	\begin{frame}
		\vfill
		\centering
		\begin{beamercolorbox}[sep=8pt,center,shadow=true,rounded=true]{title}
			\usebeamerfont{title}\insertsectionhead\par%
		\end{beamercolorbox}
		\vfill
	\end{frame}
}

\title{Quantum Speedup on Exhaustive-search Attacks on Cryptosystems}
\subtitle{Week 7 Report for Class 75}

\author{2017320009 Sangheon Lee\\ 2017320023 Mingyu Cho}

\date{\today}

\begin{document}
	\begin{frame}
		\titlepage
	\end{frame}
	
	\begin{frame}
		\frametitle{Table of Contents}
		\tableofcontents
	\end{frame}
	
	\ifproposal
	\section{Necessity}
	
	\begin{frame}
		\frametitle{Necessity}
		
		\begin{itemize}
			\item ``Quantum Supremacy" imminent
			\item Post-quantum stage makes most cryptosystems used nowadays unsafe
			\begin{itemize}
				\item Prime factorization broken by Shor's Algorithm
			\end{itemize}
		\end{itemize}
	\end{frame}
	
	\section{Content}
	
	\begin{frame}
		\frametitle{Content}
		
		\begin{itemize}
			\item Targeting symmetric-key cryptosystem
			\begin{itemize}
				\item Specifically, DES will be used.
				\item If possible, make it able to be modified for any symmetric-key cryptosystems.
			\end{itemize}
			\item Brute-force exhaustive key search quantum algorithm
		\end{itemize}
	\end{frame}
	
	\section{Methods}
	
	\begin{frame}
		\frametitle{Methods}
		\begin{itemize}
			\item Modify Grover's Algorithm to crack DES
			\begin{itemize}
				\item $\Omega(\sqrt{n})$ search algorithm for a data table of $n$
			\end{itemize}
			\item Optionally, simulate it using projectq, a python-based quantum simulator
			\item Computationally determine and verify expected run-time, compare with existing attacks.
			\item Speed up if possible and/or necessary
		\end{itemize}
	\end{frame}
	
	\section{Expectations}
	
	\begin{frame}
		\frametitle{Expectations}
		
		\begin{itemize}
			\item The quantum algorithm will prove to be faster than at least the traditional brute-force exhaustive search.
			\item Planning to contrast two methods using the length of the key as a fixed variable if the algorithm is modifiable to others.
		\end{itemize}
	\end{frame}
	
	\section{Requirements}
	
	\begin{frame}
		\frametitle{Requirements}
		
		\begin{itemize}
			\item Papers and/or textbooks on quantum computation
			\item Help from Prof. Hong(He suggested the topic)
			\item[]
			\item Currently working with Sangheon Lee to study the backgrounds
			\item Agreed to study together before parting ways
		\end{itemize}
	\end{frame}
	\fi
	
	\section{Backgrounds}
	
	\begin{frame}
		\frametitle{Mathematical Backgrounds}
		\begin{itemize}
			\item Complex numbers
			\begin{itemize}
				\item Complex plane $a+b\iu=(a,b)$
				\item Complex polar $\rho e^{\phi \iu}=\rho \cos \phi + \rho \sin \phi \iu$
			\end{itemize}
			\item 2-dimensional Hilbert space
			\begin{itemize}
				\item A complex vector space, utilizing conjugate transposes.
				\item When constricted to $\mathbb{R}$, same as $\mathbb{R}^2$ vector space.
			\end{itemize}
			\item Most of the time on Week 4 was spent on understanding the mathematical backgrounds on 2-dimensional Hilbert spaces and special matrices which can only be considered on $\mathbb{C}$.
			\item Additionally, analyzing Grover's Algorithm implementation in Microsoft Q\# was done.
		\end{itemize}
	\end{frame}
	
	\section{Qubits}
	\begin{frame}
		\frametitle{Qubits}
		\begin{itemize}
			\item Quantum Bit(Qubit for short) is a probabilistic vector of information.
			\item $\ket{0}=\begin{pmatrix} 1 \\ 0 \end{pmatrix}$, $\ket{1}=\begin{pmatrix} 0 \\ 1 \end{pmatrix}$
			\item A general qubit is represented as $u=\alpha \ket{0}+\beta \ket{1}$
		\end{itemize}
	\end{frame}
	
	\begin{frame}
		\frametitle{Three Main Types of Operation}
		\begin{itemize}
			\item Creation
			\item Reversible Operation
			\item Measurement
		\end{itemize}
	\end{frame}
	
	\begin{frame}
		\frametitle{Creation}
		\begin{itemize}
			\item As simple as creating $\ket{0}$ or $\ket{1}$.
		\end{itemize}
	\end{frame}
	
	\begin{frame}
		\frametitle{Reversible Operation}
		\begin{itemize}
			\item Represented using unitary matrix
			\begin{itemize}
				\item $U^*U=UU^*=I$, where $U^*$ is a conjugate transpose of $U$
			\end{itemize}
			\item Two frequently used operations
			\begin{itemize}
				\item X-gate $X=\begin{pmatrix} 0 & 1 \\ 1 & 0 \end{pmatrix}$ (so-called NOT gate)
				\item Hadamard Gate $H=\dfrac{1}{\sqrt{2}}\begin{pmatrix} 1 & 1 \\ 1 & -1 \end{pmatrix}$
				\item Hadamard Gate creates a Quantum Superposition
			\end{itemize}
		\end{itemize}
	\end{frame}
	
	\begin{frame}
		\frametitle{Measurement}
		\begin{itemize}
			\item A non-deterministic(probabilistic) measure of a qubit $u$.
			\item For a vector $u=\begin{pmatrix} \alpha \\ \beta \end{pmatrix}$ where $\|u\|=1$
			\begin{itemize}
				\item returns "true" with probability $\|\alpha\|^2$, and $u$ becomes $\ket{0}$
				\item returns "false" with probability $\|\beta\|^2$, and $u$ becomes $\ket{1}$
			\end{itemize}
			\item Destroys quantum superposition; often called "destructive".
		\end{itemize}
	\end{frame}
	
	\section{Grover's Algorithm}
	
	\begin{frame}{Overview}
		\begin{itemize}
			\item Grover's algorithm can find a specific state satisfying some condition among $ N = 2^n$ candidates in $ O(\sqrt{N}) $ time, compared to classical runtime complexity $ O(N) $.
			\item Grover's algorithm exploits qualities of quantum amplitudes to gain advantage of probability seperation.
			\item It can brute-force 128-bit symmetric cryptographic key in roughly $ 2^{64} $ iterations.
		\end{itemize}
	\end{frame}
	
	\begin{frame}{Algorithm}
		Input:
		\begin{itemize}
			\item A quantum oracle $ \mathcal{O} $ which performs the operation $ \mathcal{O} \ket{x}  = (-1)^{f(x)} \ket{x}$, where $ f(x) = 0 $ for all $0 \leq x < 2^n $ except $ x_0 $, for which $ f(x_0) = 1 $.
			\begin{itemize}
				\item Such quantum oracle is viable, and takes $ O(1) $ time.
			\end{itemize}
			\item $ n $ qubits initialized to the state $ \ket{0} $
		\end{itemize}
		Output: $ x_0 $, in runtime $ O(\sqrt{2^n}) $ with error rate $ O(\frac{1}{2^n}) $
	\end{frame}
	
	\begin{frame}{Algorithm}
		Procedure:
		\begin{enumerate}
			\item $ \ket{0}^{\otimes n} $ (initial state)
			\item $ H^{\otimes n} \ket{0}^{\otimes n} = \dfrac{1}{\sqrt{2^n}} \sum\limits_{x=0}^{2^n-1}\ket{x} = \ket{\psi}$ (Hadamard transform)
			\item $ [(2 \ket{\psi} \bra{\psi} - I) \mathcal{O}]^R \ket{\psi} \approx \ket{x_0}  $ (Grover iteration for $ R \approx  \frac{\pi}{4}\sqrt{2^n} $ times)
			\item $ x_0 $ (measure)
		\end{enumerate}
		Grover iteration in a nutshell: negate the amplitude of the desired state, followed by `diffusion transform' which increases the amplitude of the desired state and lower the others.
	\end{frame}
	
	\section{Quantum S-DES Oracle}
	
	\begin{frame}{S-DES}
		\begin{itemize}
			\item Simplified DES with 2 rounds
			\item Structure similar to DES but simplified with 10-bit key and 8-bit plaintext.
			\item Quantum oracle needs to be reversible, of which S-DES is (normally) not.
		\end{itemize}
	\end{frame}
	
	\begin{frame}{Quantumizing S-DES}
		\begin{itemize}
			\item Most parts are permutations or compressions; no problems here.
			\item Two parts poses a challenge:
		\end{itemize}
		\begin{enumerate}
			\item S-Boxes
			\begin{itemize}
				\item Consists of lookup table: not ideal for our situation.
				\item Broken the S-Boxes into fundamental and/or/not-gates
				\item Classic gates are not reversible; use CNOT/CCNOT gates to make them reversible.
			\end{itemize}
			\item Expansion
			\begin{itemize}
				\item Due to \href{https://en.wikipedia.org/wiki/No-cloning_theorem}{No-cloning theorem}, directly copying a qubit is not possible.
				\item Use XOR operation to copy the information.
			\end{itemize}
		\end{enumerate}
	\end{frame}
	
	\section{Current Works and Future Plans}
	
	\begin{frame}
		\frametitle{Work in Progress(Mingyu Cho)}
		\begin{itemize}
			\item Read \href{https://arxiv.org/pdf/1512.04965.pdf}{a paper on AES quantum oracle}
			\item Basic Assumption: a valid encryption pair(of 1-block length) is given, and the target is to gain the key.
			\item Theoretical Quantum conversion of S-DES completed.
			\item Implementing a non-quantum S-DES for testing the quantum version.
			\item Waiting for the code for analysis.
		\end{itemize}
	\end{frame}
	
	\begin{frame}
		\frametitle{Work in Progress(Sangheon Lee)}
		\begin{itemize}
            \item Still learning Microsoft Q\# syntax such as (implicit) closure
			\item Learn how to construct an oracle in Grover's (by phase kickback trick)
            \item Devise classic logical gates(AND, OR, XOR) in quantum computing via CCNOT gate (a.k.a Toffoli gate)
            \item Understand S-DES fundamentals
		\end{itemize}
	\end{frame}
	
	\begin{frame}
		\frametitle{Future Plans}
		\begin{itemize}
			%\item Dig in deeper on the backgrounds, if necessary.
			%\item Understand Grover's Algorithm and other quantum algorithms to know how to utilize it.
			%\item Update sample code of Grover's Algorithm by implementing oracle function and randomization.
			%\item Possibly utilize IBM Quantum Experience?
			%\item Search some more sample codes (not sure if implementing and debugging Shor's and QFT is necessary).
			\item Construct and test S-DES and apply Grover's.
            \begin{itemize}
                \item Since S-DES consists of various components, all of these must be implemented carefully and in order
            \end{itemize}
            \item Hope to import Microsoft Q\# implementation to qiskit (IBM Quantum Experience)
            \item Reduce complexity of quantum gates if possible (as Prof. Hong suggested)
		\end{itemize}
	\end{frame}
\end{document}